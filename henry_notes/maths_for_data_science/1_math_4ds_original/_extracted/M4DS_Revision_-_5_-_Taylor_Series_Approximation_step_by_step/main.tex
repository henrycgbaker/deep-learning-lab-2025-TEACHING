\documentclass[a4paper,11pt]{article} 

\usepackage[top = 2.5cm, bottom = 2.5cm, left = 2.5cm, right = 2.5cm]{geometry} 
\usepackage[T1]{fontenc}
\usepackage[utf8]{inputenc}
\usepackage{multirow} 
\usepackage{booktabs}
\usepackage{graphicx} 
\usepackage{setspace}
\setlength{\parindent}{0in}
\usepackage{enumerate}
\usepackage{float}
\usepackage{fancyhdr}
\usepackage{amsmath}
\usepackage{tcolorbox}

\DeclareMathOperator*{\argmin}{arg\,min}


\pagestyle{fancy} 
\fancyhf{}
\lhead{\footnotesize M4DS Mid-term revision}
\rhead{\footnotesize Henry Baker} 
\cfoot{\footnotesize \thepage} 


\begin{document}


\thispagestyle{empty}
\begin{tabular}{p{15.5cm}} 
{\large \bf Mid Terms Fall 2023  \\ Henry Baker}
\hline 
\\
\end{tabular} 

\vspace*{0.3cm} 
\begin{center} 
	{\Large \bf M4DS Mid Terms Revision: \\ Talor Series Approximation \\ A step-by-step how to}
	\vspace{2mm}
	
\end{center}  
\vspace{0.4cm}

\begin{tcolorbox}
\textbf{Taylor Series Approximation}
\begin{enumerate}
    \item Find derivatives ($n$ many, depending on polynonmial degree specified)
    \item Evaluate them (a = \{x\})
    \item Insert each of these into the Taylor formula...
    \item ...Simultaneously: plug in $a$ values with the given centring coordinate. This will leave various $x$ values: this your line formula.
\end{enumerate}
\end{tcolorbox}

\section{Generic}
To approximate a function \( f(x) \) using a Taylor polynomial of degree \( n \) centered at \( a \), follow these steps:

\begin{enumerate}
    \item Start with the function \( f(x) \).

    \item For each integer \( k \) from 0 to \( n \), (with $n$ being the Polynomial degree to which you are approximating):
    \begin{enumerate}
        \item Differentiate \( f(x) \) \( k \) times to get the \( k \)-th derivative, denoted as \( f^{(k)}(x) \).
        
        \item Evaluate \( f^{(k)}(x) \) at \( x = a \) to get \( f^{(k)}(a) \).
    \end{enumerate}

    \item Construct the Taylor polynomial \( P_n(x) \) using the formula:
    \[ P_n(x) = \sum_{k=0}^{n} \frac{f^{(k)}(a)}{k!}(x-a)^k \]
    This sum includes terms for each derivative from 0 to \( n \), divided by the factorial of the degree of the derivative, multiplied by the power of \( (x-a) \) corresponding to the degree of the derivative.
\end{enumerate}

The resulting polynomial \( P_n(x) \) is the Taylor polynomial of degree \( n \) approximating \( f(x) \) centered at \( a \).

\section{Example: Talor Polynomial construction for  $f(x) = 1 + x + x^2$, centred at $x = 1$}

\subsection{Talylor Series formula}

ALL THE BELOW EXAMPLES NEEED REWORKING - COME BACK TO\\
\\
Use the formula for the Taylor series:

\[f(x) \approx f(a) + f'(a)(x-a) + \frac{f''(a)}{2!}(x-a)^2 + \frac{f'''(a)}{3!}(x-a)^3 + \dots \]

We will be approximating using a quadratic ($n=2$) Taylor series:
\[ f(a) + f'(a)(x-a) + \frac{f''(a)}{2!}(x-a)^2 \] \\

\subsection{Calculate derivatives $\rightarrow$ evaluating at $x=1$}
\begin{enumerate}
    \item derive
    \item plug in $a = 1$
\end{enumerate}

\textbf{F(x):}\\
\begin{align}
    f(x) &= 1 + x + x^2 \\
    f(1) &= 1 + 1 + 1^2 = 3 
    \end{align}

\textbf{F'(x):}
\begin{align}
    f'(x) &= \frac{d}{dx}(1 + x + x^2) \\
    &= 1 + 2x \\
    f'(1) &= 1 + 2(1) \\
    &= 3
\end{align}

\textbf{F''(x):}
\begin{align}
    f''(x) &= \frac{d^2}{dx^2}(1 + x + x^2) \\
    & = 2 \\
    f''(1) &= 2 
\end{align}

\subsection{Plugging these values into the Taylor series}
In our case, the formula for a polynomial of degree two:

\[ P_2(x) = 3 + 3(x-1) + \frac{2}{2}(x-1)^2 \]
Simplifying further:
\[ P_2(x) = 3 + 3x - 3 + x^2 - 2x + 1 \]
So:
\[ P_2(x) = x^2 + x + 1 \]

Thus, the Taylor polynomial of degree two approximating \( f(x) = 1 + x + x^2 \) centered at \( a = 1 \) is:

\[ P_2(x) = x^2 + x + 1 \]




\section{Example: Talor Polynomial construction for  $f(x) = 1 + x + x^2$, centred at $x = 1$ DELETE}

Given the function:
\[ f(x) = 1 + x + x^2 \]
We want to find the Taylor polynomial of degree two centered at \( a = 1 \). \\
\begin{enumerate}
    \item \textbf{Evaluating the function at \( a = 1 \)}:
    \[ f(1) = 1 + 1 + 1^2 = 3 \]

    \item \textbf{Differentiating \( f \) with respect to \( x \):}
    \[ f'(x) = \frac{d}{dx}(1 + x + x^2) = 1 + 2x \]
    Evaluating \( f' \) at \( a = 1 \):
    \[ f'(1) = 1 + 2(1) = 3 \]
    
    \item \textbf{Differentiating \( f' \) to get the second derivative:}
    \[ f''(x) = \frac{d^2}{dx^2}(1 + x + x^2) = 2 \]
    Evaluating \( f'' \) at \( a = 1 \):
    \[ f''(1) = 2 \]

    Using the Taylor polynomial formula for degree two:
    \[ P_2(x) = f(1) + f'(1)(x-1) + \frac{f''(1)}{2!}(x-1)^2 \]
    \[ P_2(x) = 3 + 3(x-1) + (x-1)^2 \]
    \[ P_2(x) = 3 + 3x - 3 + x^2 - 2x + 1 \]
    \[ P_2(x) = x^2 + x + 1 \]
    
    Thus, the Taylor polynomial of degree two for \( f(x) = 1 + x + x^2 \) centered at \( a = 1 \) is:
    \[ P_2(x) = x^2 + x + 1 \]
\end{enumerate}

\end{document}


\end{document}