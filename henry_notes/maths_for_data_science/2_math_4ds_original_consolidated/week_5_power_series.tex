% =============================================================================
% Week 5: Power Series
% =============================================================================

\chapter{Power Series}
\label{ch:week5}

\thispagestyle{empty}
{\large \textbf{Mid Terms Fall 2023  --  Henry Baker}}
\par\noindent\rule{\textwidth}{0.4pt} 

\vspace*{0.3cm} 
\begin{center} 
	{\Large \bf M4DS Mid Terms Revision: Session 5\\ Power Series}
	\vspace{2mm}
	
\end{center}  
\vspace{0.4cm}

\section{Taylor Series Approximation}
We have an ugly function: difficult to differentiate, integrate, manipulate, maximise. \\
Assume: it has a power series representation at some point $x = a$, ie:
\[p(x) = f(x) \text{ at } x = a\]
Recall, polynomial function p(x) can be written as:
\[p(x) = \sum_{j=0}^{\infty} c_j (x - a)^j\]
Expanded:
\[p(x) = c_0 + c_1x + c_2x^2 + \dots\] \\

So, setting it to $x = 0$:\\
if we want $p(x)$ to represent $f(x)$ at $x = 0$, what conditions would we like to hold?
\begin{align}
    p(0) &= f(0) \\
    p'(0) &= f'(0) \\
    p''(0) &= f''(0) \\
    p'''(0) &= f'''(0) \\
    \dots
\end{align}
We can solve for the values of $c$ in the Talor series that make these conditions hold:
\textit{the basic intuition is that as we continue to differentate the power series $n$ times, the $x$ in the $nth$ term drop out leaving an associated constant; when we set $x = 0$ all the other terms drop out, this gives us the constant value for the $nth$ term in the power series} \\

\subsection{First Condition $p(0) = f(0)$}
\[p(x) = c_0 + c_1x + c_2x^2 + \dots\] \\
Setting to 0, $p(0) \rightarrow$ all $x$ terms drop out $\rightarrow = \dots c_0$. the first Constant term\\
So, first condition:
\begin{align}
    p(0) &= f(0) \\
    p(0) &= c_0  \\
    c_0 &= f(0) \\
\end{align}

\subsection{2nd Condition $p'(0) = f'(0)$}
\[p'(x) = c_1 + 2c_2x + 3c_3x^2 + 4c_4x^3 \dots\] \\
Setting to 0: $p'(0) \rightarrow$ all $x$ terms drop out $\rightarrow = \dots c_1$. the first Constant term\\
So, second condition:
\begin{align}
    p'(0) &= f'(0) \\
    p'(0) &= c_1  \\
    c_1 &= f'(0) \\
\end{align}

\subsection{Third Condition: $p''(0) = f''(0)$}
\[p''(x) = 2c_2 + 6c_3x + 12c_4x^2 \dots\] \\
Setting to 0: $p''(0) \rightarrow$ all $x$ terms drop out $\rightarrow = \dots 2c_2$. the first Constant term\\
So, third condition:
\begin{align}
    p''(0) &= f''(0) \\
    p''(0) &= 2c_2  \\
    2c_2 &= f''(0) \\
    c_2 &= \frac{1}{2}f''(0)
\end{align}

\subsection{Fourth Condition: $p'''(0) = f'''(0)$}
\[p'''(x) = 6c_3 + 24c_4x +  \dots\] \\
Setting to 0: $p'''(0) \rightarrow$ all $x$ terms drop out $\rightarrow = \dots 6c_3$. the first Constant term\\
So, fourth condition:
\begin{align}
    p'''(0) &= f'''(0) \\
    p'''(0) &= 6c_3  \\
    6c_3 &= f'''(0) \\
    c_c &= \frac{1}{6}f'''(0)
\end{align}

\subsection{Putting it Together}
We have following conditions: 
\begin{enumerate}
    \item $c_0 = f(0)$
    \item $c_1 = f'(0)$
    \item $c_2 = \frac{1}{2}f''(0)$
    \item $c_3 = \frac{1}{6}f'''(0)$
\end{enumerate}
to represent $f(x)$ at $x = 0$, our polynomial:
\[p(x) = \underbrace{f(0)}_{c_0} + \underbrace{f'(0)}_{c_1} x + \frac{1}{2} \underbrace{f''(0)}_{c_2} x^2 + \frac{1}{6} \underbrace{f'''(0)}_{c_3} x^3 + \dots\]

If you kept going, a pattern emerges:
\[p(x) = \underbrace{f(0)}_{c_0} + \underbrace{f'(0)}_{c_1} x + \frac{1}{2!} \underbrace{f''(0)}_{c_2} x^2 + \frac{1}{3!} \underbrace{f'''(0)}_{c_3} x^3 + \dots\]
Generalising to the form $x = 0$, to $x=a$
\[p(x) = \underbrace{f(a)}_{c_0} + \underbrace{f'(a)}_{c_1} (x-a) + \frac{1}{2!} \underbrace{f''(a)}_{c_2} (x-a)^2 + \frac{1}{3!} \underbrace{f'''(a)}_{c_3} (x-a)^3 + \dots\]
This is the Taylor series for the function $f (x)$ at $a$. In the special case when $a = 0$, we also call this the Maclaurin series.\\
From this, we can write the $n$th \emph{partial sum} of the Maclaurin series as: 
\begin{align*}
p_0(x) & = f(0) \\
p_1(x) & = f(0) + f'(0)x \\
p_2(x) & = f(0) + f'(0)x + \frac{1}{2} f''(0)x^2 \\
p_3(x) & = f(0) + f'(0)x + \frac{1}{2} f''(0)x^2 + \frac{1}{6} f'''(0)x^3 \\
\end{align*}
Each partial sum is an approximation to the function \( f(x) \), which gets better and better as you add more terms.

\subsection{Example:} Find the Taylor polynomials \( p_0 \), \( p_1 \), \( p_2 \), and \( p_3 \) for the function \( f(x) = \log(x) \) at \( x = 1 \).

We will require the derivatives of \( f(x) \), and we should also evaluate them at \( x = 1 \):

\begin{align*}
f(x) &= \log(x) &\rightarrow \log(1) &= 0 \\
f'(x) &= \frac{1}{x} &\rightarrow f'(1) &= 1 \\
f''(x) &= -\frac{1}{x^2} &\rightarrow f''(1) &= -1 \\
f'''(x) &= \frac{2}{x^3} &\rightarrow f'''(1) &= 2 \\
\end{align*}

\[p(x) = \underbrace{f(a)}_{c_0} + \underbrace{f'(a)}_{c_1} (x-a) + \frac{1}{2!} \underbrace{f''(a)}_{c_2} (x-a)^2 + \frac{1}{3!} \underbrace{f'''(a)}_{c_3} (x-a)^3 + \dots\]

\begin{align*}
p_0(x) &= f(1) = 0 \\
p_1(x) &= f(1) + f'(1)(x - 1) \\
&= 0 + 1(x - 1) \\
&= x - 1 \\
p_2(x) &= x - 1 + \frac{1}{2} f''(1)(x - 1)^2 \\
&= x - 1 - \frac{1}{2} (x - 1)^2 \\
p_3(x) &= x - 1 - \frac{1}{2} (x - 1)^2 + \frac{1}{3!} (x - 1)^3 \\
&= 
\end{align*}

\section{Integration}
