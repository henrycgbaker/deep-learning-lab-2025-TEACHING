% =============================================================================
% Week 1: Probability Theory
% =============================================================================

\chapter{Probability Theory}
\label{ch:week1}

\thispagestyle{empty}
{\large \textbf{Mid Terms Fall 2023 -- Henry Baker}}
\par\noindent\rule{\textwidth}{0.4pt} 

\vspace*{0.3cm} 
\begin{center} 
	{\Large \bf M4DS Mid-term Revision \\ 
 Session 1: Probability Theory}
	\vspace{2mm}
\end{center}  
\vspace{0.4cm}

\textbf{De Morgan's Law:}
\begin{align*}
(A \cup B)^c &= A^c \cap B^c \\
(A \cap B)^c &= A^c \cup B^c
\end{align*}

\textbf{Notation:}\\
Let $A_n$ be the event that the $nth$ flip is Heads
\begin{itemize}
    \item \( A_1 = \{(1, s_2, \ldots, s_{10}) : s_j \in \{0, 1\} \text{ for } 2 \leq j \leq 10\} \) : \textbf{First flip = H}
    \item \( B = \bigcup_{n=1}^{10} A_n = A_1 \cup A_2 \cup \dots \cup A_{10} \) : \textbf{Event that at least one flip was Heads}
    \item \( C = \bigcap_{n=1}^{10} A_n = A_1 \cap A_2 \cap \dots \cap A_{10} \) : \textbf{ Event that all flips were Heads}
    \item \( D = \bigcup_{n=1}^{9} (A_n \cap A_{n+1}) \) : \textbf{Event that there were at least two consecutive Heads}
\end{itemize}
\vspace{0.4cm}


\textbf{Naive Definition of Prob:}
\[P_{\text{naive}}(A) = \frac{|A|}{|S|} = \frac{\text{number of outcomes favorable to } A}{\text{total number of outcomes in } S} \]
In general, \(P_{\text{naive}}(A^c) = 1 --- P_{\text{naive}}(A)\)\\
\vspace{0.4cm}


\textbf{Multiplication Rule}: 
\begin{itemize}
    \item Think of it in terms of trees (see slide 26) 
    \item NB: doesn't matter which order the events are structured in (counter intuitive).
    \item All sampling derives from multiplication rule\\
\end{itemize}
\vspace{0.4cm}


\textbf{Combinations} = when order does not matter: think of multiplication rule \\
\textbf{permutations} = when order/position matters: \(n!\) \\

\vskip 0.4 cm

$\begin{array}{c|c|c}
 & \textbf{Order Matters} & \textbf{Order Doesn't Matter} \\
\par\noindent\rule{\textwidth}{0.4pt}
\textbf{With Replacement} & n^k & \binom{n+k-1}{k} \\
\par\noindent\rule{\textwidth}{0.4pt}
\textbf{Without Replacement} & \frac{n!}{(n-k)!} & \binom{n}{k} = \frac{n!}{k!(n-k)!} \\
\end{array}$

\vskip 0.4 cm

NB: order matters, sampling w/ replacement also written as $n\cdot (n-1) \cdot (n-2) \cdot \dots (n-k+1)$ \\
\textbf{Sampling with replacement}: \(n^k\)\\
\textbf{Sampling w/o replacement}: \(n * (n-1)*\ldots{}(n --- k + 1)\) \\
\textbf{Birthday Problem} --- counting complement\\
What’s the probability of no matching birthdays? \\
This amounts to sampling the days of the year without replacement:
\begin{align*}
    P(\text{no birthday match}) &= \frac{\text{number of ways to not repeat birthdays}}{\text{number of total possibilities}} \\
    &= \frac{365 \times 364 \times \dots \times (365 --- k + 1)}{365^k} \\
    P(\text{birthday match}) &= 1 --- \frac{365 \times 364 \times \dots \times (365 --- k + 1)}{365^k}
\end{align*}

\textbf{Leibniz' mistake:} when order matters, make sure to label items:

2 dice --> 2x ways to get 11, 1x way to get 12

\textbf{Binomial Coefficient}:
\[\binom{n}{k}\]
= number of subsets of size k for set n.

Order does not matter.
\begin{align*}
    \binom{n}{k} &= \frac{n(n-1) \dots (n-k+1)}{k!} \\
     &= \frac{n!}{(n-k)!k!}
\end{align*}
For $k > n$, we have $\binom{n}{k} = 0.$\\
PRACTICE QS: SLIDE 35\\
\textbf{Non-Naive Probabilities: Probability Functions} \\    
\begin{itemize}
    \item a probability space consisting of a \emph{sample space $S$} and a \emph{probability function $P$}
    \item which takes an event $A \subseteq S$ as input and returns $P(A)$, a real number between 0 and 1, as output. 
     \item The function P must satisfy the following two axioms:
            \begin{itemize}
                \item \( P(\emptyset) = 0, P(S) = 1 \).
                \item If \( A_1, A_2, \dots \) are disjoint events, then 
                    \[ P\left(\bigcup_{j=1}^{\infty} A_j\right) = \sum_{j=1}^{\infty} P(A_j) \].
             \end{itemize}
    \end{itemize}

\textbf{Properties of Prob (cheat sheet)} \\
        \begin{itemize}
            \item $P(A^c) = 1 --- P(A)$ \\
            \item If $A \subseteq B$, then $P(A) \leq P(B)$. \\
            \item $P(A \cup B) = P(A) + P(B) --- P(A \cap B)$.
        \end{itemize}

% \end{enumerate}  % Note: no matching \begin{enumerate} in original
