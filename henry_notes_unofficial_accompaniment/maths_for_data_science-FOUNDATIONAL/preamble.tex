% =============================================================================
% Shared Preamble for M4DS Notes - Enhanced Version
% =============================================================================

\documentclass[a4paper,11pt]{report}

% Page geometry
\usepackage[top=2.5cm, bottom=2.5cm, left=2.5cm, right=2.5cm]{geometry}

% Encoding
\usepackage[T1]{fontenc}
\usepackage[utf8]{inputenc}

% Tables
\usepackage{multirow}
\usepackage{booktabs}

% Graphics
\usepackage{graphicx}
\graphicspath{{./images/}}

% TikZ for diagrams
\usepackage{tikz}
\usetikzlibrary{trees,positioning,arrows.meta}

% Spacing
\usepackage{setspace}
\setlength{\parindent}{0pt}
\setlength{\parskip}{0.5\baselineskip plus 2pt minus 1pt}
\raggedbottom  % Prevent vertical stretching to fill pages

% Lists
\usepackage{enumerate}
\usepackage{enumitem}

% Floats
\usepackage{float}

% Headers
\usepackage{fancyhdr}
\pagestyle{fancy}
\fancyhf{}
\lhead{\footnotesize Henry Baker}
\rhead{\footnotesize Maths for Data Science - 2024 Notes for TAing Labs}
\cfoot{\footnotesize \thepage}

% Math
\usepackage{amsmath}
\usepackage{amssymb}
\usepackage{bm}
\usepackage{amsthm}

% Math operators
\DeclareMathOperator*{\argmin}{arg\,min}
\DeclareMathOperator*{\argmax}{arg\,max}
\DeclareMathOperator{\DUnif}{DUnif}
\DeclareMathOperator{\Var}{Var}
\DeclareMathOperator{\Cov}{Cov}
\DeclareMathOperator{\E}{\mathbb{E}}
\DeclareMathOperator{\Prob}{\mathbb{P}}
\DeclareMathOperator{\Tr}{Tr}
\DeclareMathOperator{\rank}{rank}
\DeclareMathOperator{\diag}{diag}
\DeclareMathOperator{\Corr}{Corr}

% centernot for negating arrows (e.g., \centernot\implies)
\usepackage{centernot}

% =============================================================================
% Theorem Environments
% =============================================================================
\theoremstyle{definition}
\newtheorem{definition}{Definition}[chapter]
\newtheorem{example}{Example}[chapter]

\theoremstyle{plain}
\newtheorem{theorem}{Theorem}[chapter]
\newtheorem{lemma}[theorem]{Lemma}
\newtheorem{proposition}[theorem]{Proposition}
\newtheorem{corollary}[theorem]{Corollary}

\theoremstyle{remark}
\newtheorem*{remark}{Remark}
\newtheorem*{note}{Note}

% =============================================================================
% Colored Boxes (tcolorbox)
% =============================================================================
\usepackage{tcolorbox}
\tcbuselibrary{breakable,skins}

% Original blue box for key formulas
\newtcolorbox{bluebox}[1][]{
    colback=blue!5!white,
    colframe=blue!75!black,
    title=#1,
    breakable
}

% Key result box (green)
\newtcolorbox{keyresult}[1][Key Result]{
    colback=green!5!white,
    colframe=green!60!black,
    title=#1,
    fonttitle=\bfseries,
    breakable
}

% Intuition box (yellow)
\newtcolorbox{intuition}[1][Intuition]{
    colback=yellow!5!white,
    colframe=yellow!60!black,
    title=#1,
    fonttitle=\bfseries,
    breakable
}

% Warning box (red)
\newtcolorbox{warning}[1][Warning]{
    colback=red!5!white,
    colframe=red!60!black,
    title=#1,
    fonttitle=\bfseries,
    breakable
}

% Rigour box (grey) - for formal definitions, proofs, derivations
\newtcolorbox{rigour}[1][]{
    colback=gray!8!white,
    colframe=gray!60!black,
    title=#1,
    fonttitle=\bfseries,
    breakable,
    before upper={\parskip=0.5\baselineskip}
}

% Proof environment styling
\renewcommand{\qedsymbol}{$\blacksquare$}

% =============================================================================
% Hyperlinks (load before cleveref)
% =============================================================================
\usepackage{hyperref}
\hypersetup{
    colorlinks=true,
    linkcolor=blue!70!black,
    citecolor=green!50!black,
    urlcolor=blue!70!black
}

% =============================================================================
% Cross-referencing (must load after hyperref)
% =============================================================================
\usepackage{cleveref}
\crefname{theorem}{Theorem}{Theorems}
\crefname{lemma}{Lemma}{Lemmas}
\crefname{definition}{Definition}{Definitions}
\crefname{example}{Example}{Examples}
\crefname{equation}{Equation}{Equations}
\crefname{figure}{Figure}{Figures}
\crefname{table}{Table}{Tables}
\crefname{chapter}{Chapter}{Chapters}
\crefname{section}{Section}{Sections}
